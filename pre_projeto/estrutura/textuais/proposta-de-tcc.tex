% PROPOSTA DE TRABALHO DE CONCLUSÃO DE CURSO-----------------------------------------------------------

\chapter{PRÉ-PROJETO DE TRABALHO DE CONCLUSÃO DE CURSO}
\label{chap:proposta}

\section{TÍTULO}
\label{sec:titulo}
% Informe o título do trabalho-------------------------------------------------------------------------
Processamento de Linguagem Natural Aplicada a Textos Jurídicos.
%------------------------------------------------------------------------------------------------------

\section{MODALIDADE DO TRABALHO}
\label{sec:modalidade}
% Indique a Modalidade do Trabalho---------------------------------------------------------------------
% Opções:
% - Pesquisa
% - Desenvolvimento de Sistemas
Desenvolvimento de aplicação.
%------------------------------------------------------------------------------------------------------

\section{ÁREA DO TRABALHO}
\label{sec:area}
% Indique a Área do Trabalho---------------------------------------------------------------------------
Inteligência Artificial; Processamento de Linguagem Natural; Análise de Sentimentos.
%------------------------------------------------------------------------------------------------------

\section{RESUMO}
\label{sec:resumo}
% Resumo do Trabalho-----------------------------------------------------------------------------------
O Projeto ``Amora'' foi concebido com o objetivo de fortalecer a representatividade das mulheres no mercado de trabalho tecnológico, por meio da criação de uma plataforma de classificação de empresas de tecnologia quanto à amigabilidade ao gênero feminino, a partir de documentos jurídicos de reclamações trabalhistas sobre questões relacionadas a gênero. A extração automatizada de informações de documentos legais é um problema relevante, e que ainda não foi completamente resolvido. Neste sentido, um dos componentes da plataforma terá a função de extrair informações de interesse, tais como conceitos jurídicos e entidades nomeadas (locais, organizações, datas e referências), de documentos jurídicos de reclamações trabalhistas obtidos por meio de \textit{crawlers}. Estas informações serão identificadas usando informações semânticas da saída de um analisador de linguagem natural, e a partir delas, os documentos serão classificados, utilizando técnicas de Análise de Sentimentos. Essa abordagem linguística é baseada nas propostas de \citeonline{Quaresma2010} e \citeonline{Lopes2012}, que obtiveram resultados promissores.
%------------------------------------------------------------------------------------------------------