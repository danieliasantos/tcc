% DESCRIÇÃO DA PROPOSTA--------------------------------------------------------------------

\chapter{DESCRIÇÃO DA PROPOSTA}
\label{chap:descricao}

\section{INTRODUÇÃO}
\label{sec:introducao}
% Introdução-------------------------------------------------------------------------------
% (máximo de 1 página)
O Centro Integrado de Operações da Prefeitura de Belo Horizonte (COP-BH), é a entidade municipal responsável pela integração de informações e da atuação das Instituições envolvidas na resposta a problemas públicos de Belo Horizonte. Suas atividades se baseiam no \textit{Modelo de Gestão Integrada do COP-BH} \cite{ModeloGestaoCOP}, que expressa o seu posicionamento institucional e como promove a integração, alinhamento e harmonização dos processos de trabalho de diversas Instituições e Agências comprometidas com o cuidado da cidade.

Pode ser classificado, segundo a Carta Brasileira para Cidades Inteligentes \cite{CartaCidades}, como um ``Centro de Gestão Integrada - GCI'', sendo este um ambiente estratégico que busca melhorar a eficiência e eficácia da prestação de serviços públicos ao aprimorar a proteção social e governança pública.

O COP-BH viabiliza a integração entre as Instituições e Agências por meio de seis linhas de atuação: \textit{Monitoramento da Cidade, Pronta Resposta, Gestão de Crises, Operações Integradas, Gestão de Eventos e Prevenção de Problemas.}

Conforme o citado modelo \cite{ModeloGestaoCOP}, a linha de atuação de Monitoramento da Cidade possui três vertentes: o monitoramento de sensores não especialistas, tais como câmeras, o monitoramento de sensores especialistas, tais como os meteorológicos, e o monitoramento de fontes de inteligência. Seu objetivo final é permitir que o COP-BH aja preventivamente a fim de evitar problemas públicos ou responder adequadamente a estes, minorando as suas consequências.

Este trabalho analisa e propõe uma mudança no método de análise de dados, bem como sua automatização, para suporte a uma das atividades da linha de atuação de Monitoramento da Cidade, relacionadas ao monitoramento de câmeras, que se delinea a seguir.

\subsection{Contextualização do Projeto}

Conforme descreve o modelo \cite{ModeloGestaoCOP}, a atividade de monitoramento se utiliza de diversos tipos de sensores para sua consecução. Dentre estes sensores, as câmeras instaladas em locais públicos permitem acesso à visualização de diversos pontos da cidade em tempo real.

A atividade de monitoramento das câmeras, ou seja, a visualização das suas imagens em tempo real, é realizada por agentes das Instituições que têm assento no ambiente chamado Sala de Controle Integrado, local onde as imagens são acessadas com o propósito de identificar problemas públicos mais rapidamente, a fim de possibilitar maior celeridade também à resposta.

São diversos os tipos de problemas públicos monitorados, tais como a deposição clandestina de lixo ou inservíveis em locais impróprios, animais de grande porte soltos em via pública, pichação, depredação, acidentes de trânsito, invasão de áreas protegidas e furtos de cabos, entre outros.

Para tornar o serviço mais eficiente, o COP-BH desenvolveu um processo de análise de dados sobre problemas públicos para elaborar um roteiro de monitoramento, que sugere os problemas que devem ser monitorados, os períodos de maior relevância para executar o monitoramento e quais as câmeras deveriam ser monitoradas. 

A partir deste roteiro, as equipes de monitoramento teriam um guia de referência que poderia aprimorar o seu trabalho no sentido da eficiência -- pois conheceriam as câmeras deveriam priorizar no monitoramento e quando deveriam ser monitoradas, e da eficácia -- pois isso poderia aumentar a probabilidade de visualizar algum problema público durante o monitoramento.

Para fins deste trabalho, o problema público analisado é o furto de cabos. Para isso, foram utilizados dados reais de ocorrências públicas deste problema em Belo Horizonte, e são originados do Sistema de Gestão de Ocorrências Integradas (SICOP Ocorrências), por meio do qual as Instituições e Agências compartilham informações no COP-BH, Sistema de Gestão da Operação (SGO) utilizado pela BHTrans para controle de suas operações, e dados de furtos de cabos de telefonia compartilhados pelas operadoras de telefonia Oi, Claro, Vivo e V.tal. Estes dados são os mesmos utilizados na elaboração do roteiro de monitoramento. O dataset utilizado para este trabalho contém dados do período de 03/2018 a 05/2025, e consiste dos seguintes atributos:

\begin{itemize}
  \item{\texttt{origem}, que apresenta de qual sistema os dados se originaram;}
  \item{\texttt{data\_hora}, que apresenta a data e hora em que o furto ocorreu;}
  \item{\texttt{mes\_ano}, que apresenta o mês e ano da ocorrência (dados adicionados para facilitar a análise);}
  \item{\texttt{trimestre\_ano}, que apresenta o trimestre e ano da ocorrência (dados adicionados para facilitar a análise);}
  \item{\texttt{latitude} e \texttt{longitude}, que apresenta a localização exata da ocorrência; e}
  \item{\texttt{endereco} e \texttt{regional}, que apresenta o endereço físico aproximado da ocorrência.}
\end{itemize}

\subsection{Definição do Problema}

Atualmente, o COP-BH elabora o roteiro de monitoramento agrupando as ocorrências pela sua localização, e utiliza o método de estimativa de densidade de kernel para gerar um mapa de calor geoespacial (\textit{heatmap}) \cite{Wilkinson2009} para demonstrar visualmente as regiões da cidade de Belo Horizonte onde há maior concentração de furtos, utilizando a ferramenta de sistema de informação geográfica QGIS \cite{Qgis}. Então, dados de localização das câmeras públicas da cidade são sobrepostos ao mapa de calor, sendo este o resultado final do roteiro de monitoramento, conforme demonstrado na Figura \ref{fig:mapa_roteiro}.

\begin{figure}[!htb]
  \captionsetup{singlelinecheck=false}
  \centering
  \includegraphics[scale=1,keepaspectratio]{dados/images/mapa_roteiro.png}
  \caption{Roteiro de Monitoramento: Mapa de calor de locais de furtos de cabos em Belo Horizonte, com sobreposição de câmeras em locais públicos.}
  \fonte{Produção própria}
  \label{fig:mapa_roteiro}
\end{figure}

Contudo, 

\subsection{Relevância do Problema}

\subsection{Justificativa}

\subsection{Desafios do Projeto}

\subsection{Contribuição}

%-----------------------------------------------------------------------------------------

\section{OBJETIVOS}
\label{sec:objetivos}
% (máximo de 1/2 página)

\subsection{Objetivo Geral}
\label{subsec:objgeral}
% Objetivo Geral--------------------------------------------------------------------------
O objetivo geral é tratado em seu sentido mais amplo e constitui a ação que conduzirá ao tratamento da questão abordada no problema de pesquisa, fazendo menção ao objeto de uma forma mais direta. O objetivo geral deve:

Conter descrição do que vai fazer, de forma precisa e objetiva;

Ser diretamente ligado ao título;

Ser mais detalhado que o título;

Resolver o problema proposto;

Ser Claro, Conciso e Completo (CCC) e deve ser verificável ao final do trabalho.

(substitua este texto pelo objetivo geral do trabalho)
%-----------------------------------------------------------------------------------------

\subsection{Objetivos Específicos}
\label{subsec:objespc}
% Objetivos Específicos-------------------------------------------------------------------
Os objetivos específicos apresentam, de forma pormenorizada, detalhada, as ações que se prentede alcançar e estabelecem estreita relação com as particularidades relativas à temática trabalhada. Os objetivos específicos devem:

Ser Claros, Concisos e Completos (CCC) e devem ser verificáveis ao final do trabalho;

Fazem parte do detalhamento do objetivo geral ;

Devem ser iniciados com o verbo no infinitivo;

Podem ser considerados com subprodutos do objetivo geral.

(substitua este texto pelos objetivos específicos do trabalho)
%-----------------------------------------------------------------------------------------

\section{TRABALHOS CORRELATOS}
\label{sec:estadoarte}
% Estado da Arte--------------------------------------------------------------------------
% (máximo de 2 páginas)
Uma vez formulado o problema a ser atacado, é preciso se inteirar do que já foi feito, dito e discutido sobre ele. Pode ser que a dúvida, que está motivando a pesquisa, já tenha sido respondida de alguma maneira por alguém. Por isso, é preciso aprofundar o conhecimento sobre a questão, antes de dar prosseguimento ao projeto.

Essa etapa também recebe o nome de revisão bibliográfica, quando são estudados os trabalhos que se situam na circunvizinhança do problema, trabalhos que versam sobre problemas similares.

Vê-se aí por que a revisão bibliográfica é importante. De um lado, ela deve comprovar que o pesquisador não está querendo realizar algo que já foi feito, de outro lado, ela ajuda a encaminhar o passo seguinte da pesquisa, a justificativa, quer dizer, a argumentação sobre a relevância do trabalho.

Para a proposta de TCC deve ser descrito, de maneira breve, alguns (sugestão de 2 (dois) a 3 (três)) trabalhos correlatos, converse com seu orientador para citar os mais relevantes do tema abordado. Pode ser seguido a seguinte sugestão de parágrafos/tópicos:

P1. Descrição do trabalho 1

P2. Descrição do trabalho 2

P3. Descrição do trabalho 3

P4. Discussão dos trabalhos mencionados destacando porque eles são importantes para o trabalho proposto.

Para utilização de citações atente ao tipo de citação que se deseja usar. As citações são classificadas em indeireta e direta, podem ser longas ou curtas.

Uma citação indireta é a transcrição, com suas próprias palavras, das idéias de um autor, mantendo-se o sentido original. A citação indireta é a maneira que o pesquisador tem de ler, compreender e gerar conhecimento a partir do conhecimento de outros autores. Quanto à chamada da referência, ela pode ser feita de duas maneiras distintas, conforme o nome do(s) autor(es) façam parte do seu texto ou não. Exemplo de chamada fazendo parte do texto:\\
\\Enquanto \citeonline{Maturana2003} defendem uma epistemologia baseada na biologia. Para os autores, é necessário rever \ldots.\\

A chamada de referência foi feita com o comando \verb|\citeonline{chave}|, que produzirá a formatação correta.

A segunda forma de fazer uma chamada de referência deve ser utilizada quando se quer evitar uma interrupção na sequência do texto, o que poderia, eventualmente, prejudicar a leitura. Assim, a citação é feita e imediatamente após a obra referenciada deve ser colocada entre parênteses. Porém, neste caso específico, o nome do autor deve vir em caixa alta, seguido do ano da publicação. Exemplo de chamada não fazendo parte do texto:\\
\\Há defensores da epistemologia baseada na biologia que argumentam em favor da necessidade de \ldots \cite{Maturana2003}.\\

Nesse caso a chamada de referência deve ser feita com o comando \verb|\cite{chave}|, que produzirá a formatação correta.

Uma citação direta é a transcrição ou cópia de um parágrafo, de uma frase, de parte dela ou de uma expressão, usando exatamente as mesmas palavras adotadas pelo autor do trabalho consultado.

Quanto à chamada da referência, ela pode ser feita de qualquer das duas maneiras, assim como nas nas citações indiretas, conforme o nome do(s) autor(es) façam parte do texto ou não. Há duas maneiras distintas de se fazer uma citação direta, conforme o trecho citado seja longo ou curto.

Quando o trecho citado é longo (4 ou mais linhas) deve-se usar um parágrafo específico para a citação, na forma de um texto recuado (4 cm da margem esquerda), com tamanho de letra menor e espaçamento entrelinhas simples. Exemplo de citação longa:
\\\begin{citacao}
Desse modo, opera-se uma ruptura decisiva entre a reflexividade filosófica, isto é a possibilidade do sujeito de pensar e de refletir, e a objetividade científica. Encontramo-nos num ponto em que o conhecimento científico está sem consciência. Sem consciência moral, sem consciência reflexiva e também subjetiva. Cada vez mais o desenvolvimento extraordinário do conhecimento científico vai tornar menos praticável a própria possibilidade de reflexão do sujeito sobre a sua pesquisa \cite[p.~28]{Silva2000}.
\end{citacao}

Para fazer a citação longa deve-se utilizar os seguintes comandos:
\begin{verbatim}
\begin{citacao}
<texto da citacao>
\end{citacao}
\end{verbatim}

No exemplo acima, para a chamada da referência o comando \verb|\cite[p.~28]{Silva2000}| foi utilizado, visto que os nomes dos autores não são parte do trecho citado. É necessário também indicar o número da página da obra citada que contém o trecho citado.

Quando o trecho citado é curto (3 ou menos linhas) ele deve inserido diretamente no texto entre aspas. Exemplos de citação curta:\\
\\A epistemologia baseada na biologia parte do princípio de que "assumo que não posso fazer referência a entidades independentes de mim para construir meu explicar" \cite[p.~35]{Maturana2003}.\\
\\A epistemologia baseada na biologia de \citeonline[p.~35]{Maturana2003} parte do princípio de que "assumo que não posso fazer referência a entidades independentes de mim para construir meu explicar".\\

Outros exemplos de comandos para as chamadas de referências e o resultado produzido por estes são:\\
\\\citeonline{Maturana2003} \ \ \  \verb|\citeonline{Maturana2003}|\\
\citeonline{Barbosa2004} \ \ \   \verb|\citeonline{Barbosa2004}|\\
\cite[p.~28]{Silva2000} \ \ \  \verb|\cite[p.~28]{Silva2000}|\\
\citeonline[p.~33]{Silva2000} \ \ \   \verb|\citeonline[p.~33]{v}|\\
\cite[p.~35]{Maturana2003} \ \ \   \verb|\cite[p.~35]{Maturana2003}|\\
\citeonline[p.~35]{Maturana2003} \ \ \   \verb|\citeonline[p.~35]{Maturana2003}|\\
\cite{Barbosa2004,Maturana2003} \ \ \   \verb|\cite{Barbosa2004,Maturana2003}|\\

Em relação as referências, a bibliografia é feita no padrão \textsc{Bib}\TeX{}. As referências são colocadas em um arquivo separado. Neste template as referências são armazenadas no arquivo "base-referencias.bib".

Existem diversas categorias documentos e materiais componentes da bibliografia. A classe abn\TeX{} define as seguintes categorias (entradas):

\begin{verbatim}
@book
@inbook
@article
@phdthesis
@mastersthesis
@monography
@techreport
@manual
@proceedings
@inproceedings
@journalpart
@booklet
@patent
@unpublished
@misc
\end{verbatim}

Cada categoria (entrada) é formatada pelo pacote \citeonline{abnTeX22014d} de uma forma específica. Para maiores detalhes, refira-se a \citeonline{abnTeX22014d}, \citeonline{abnTeX22014b}, \citeonline{abnTeX22014c}.
%-----------------------------------------------------------------------------------------

\section{METODOLOGIA/METODOLOGIA DE DESENVOLVIMENTO} % Escolher o nome mais adequado ao trabalho
\label{sec:metodologia}
% Procedimentos Metodológicos/Metodologia-------------------------------------------------
% (máximo de 2 páginas)
Nesta seção (ver qual o nome mais adequado ao trabalho) deve ser descrito sucintamente o procedimentos metodológicos para a execução do projeto ressaltando como os objetivos serão alcançados.

Em geral, a seção descreve os procedimentos usados para resolver o problema atacado. Pode ser estruturada em tópicos, onde cada tópico representa um subproduto do objetivo geral.

No caso de desenvolvimento de sistemas deve-se descrever a metodologia a ser utilizada, por exemplo Scrum, eXtreme Programming, RUP, etc.

Também pode ser descritos técnicas de desenvolvimento de software como por exemplo TDD, BDD, SPA,  etc.

% http://latexbr.blogspot.com.br/2011/07/inserindo-figuras-no-latex.html
\begin{figure}[!htb]
  \captionsetup{singlelinecheck=false}
  \centering
  \begin{measuredfigure}
    \includegraphics[scale=0.5,keepaspectratio]{dados/images/latex-logo.png}
    \caption{Logo do Latex}
    \fonte{\citeonline{abnTeX22014d}}
  \end{measuredfigure}
  \label{fig:latex_logo}
\end{figure}

(substitua este texto pelo de procedimentos metodológicos/metodologia do trabalho)

%-----------------------------------------------------------------------------------------

\section{CONCLUSÃO/CONSIDERAÇÕES FINAIS} % Escolher o nome mais adequado ao trabalho
\label{sec:conclusao}
% Conclusão/Considerações Finais----------------------------------------------------------
% (máximo de ½ página)
Na seção de Conclusão ou Considerações Finais (ver qual o nome mais adequado ao trabalho) o acadêmico deve descrever:

Como espera alcançar os objetivos propostos;

Destacar as dificuldades encontradas e previstas;

Fazer o fechamento do trabalho destacando sua importância.

(substitua este texto pelo de estado da arte do trabalho)
%-----------------------------------------------------------------------------------------

\section{PLANEJAMENTO DO TRABALHO}
\label{sec:planejamento}
% Planejamento do Trabalho----------------------------------------------------------------
% Esta seção não precisa ser editada, apenas edite o quadro 1 armazenada no diretório ".\dados\quadros"
O planejamento do trabalho de estágio que será desenvolvido pelo aluno, ao longo do período letivo, está descrito no cronograma da Quadro 1. Neste cronograma constam todas as atividades com seus respectivos prazos para o cumprimento.
\input{./dados/quadros/quadro1}

\subsection{DA PROPOSTA AO PROJETO}
Nesta seção o acadêmico deve descrever o que será desenvolvido até o projeto de TCC, ou seja,
o que irá entrega no projeto de TCC, além do que já foi feito na proposta.

%-----------------------------------------------------------------------------------------

\section{RECURSOS NECESSÁRIOS}
\label{sec:recursos}
% Recursos Necessários--------------------------------------------------------------------
Coloque todos os materiais que serão utilizados. Exemplos: computadores, equipamentos de redes, licenças de software, etc. Também deverá ser colocado se os recursos estarão disponíveis. A universidade não comprará os recursos, portanto a responsabilidade de comprar algo será do aluno. (substitua este texto pelo de recursos necessários do trabalho)
%-----------------------------------------------------------------------------------------

\section{HORÁRIO DE TRABALHO}
\label{sec:horário}
% Horário de Trabalho---------------------------------------------------------------------
% Esta seção não precisa ser editada, apenas edite o quadro 2 armazenada no diretório ".\dados\quadros"
O horário destinado para realização das atividades do TCC, bem como o horário destinado para a reunião semanal/quinzenal com o orientador estão descritos no cronograma do Quadro 2. Este horário é definido com orientador levando em consideração a complexidade do trabalho a ser desenvolvido.
\input{./dados/quadros/quadro2}
%-----------------------------------------------------------------------------------------
