% DESCRIÇÃO DA PROPOSTA--------------------------------------------------------------------

\chapter{DESCRIÇÃO DA PROPOSTA}
\label{chap:descricao}

\section{INTRODUÇÃO}
\label{sec:introducao}
% Introdução-------------------------------------------------------------------------------
O Centro Integrado de Operações da Prefeitura de Belo Horizonte (COP-BH), é a entidade municipal responsável pela integração de informações e da atuação das Instituições envolvidas na resposta a problemas públicos de Belo Horizonte. Suas atividades se baseiam no \textit{Modelo de Gestão Integrada do COP-BH} \cite{ModeloGestaoCOP}, que expressa o seu posicionamento institucional e como promove a integração, alinhamento e harmonização dos processos de trabalho de diversas Instituições e Agências comprometidas com o cuidado da cidade.

Pode ser classificado, segundo a Carta Brasileira para Cidades Inteligentes \cite{CartaCidades}, como um ``Centro de Gestão Integrada - GCI'', sendo este um ambiente estratégico que busca melhorar a eficiência e eficácia da prestação de serviços públicos ao aprimorar a proteção social e governança pública.

O COP-BH viabiliza a integração entre as Instituições e Agências por meio de seis linhas de atuação: \textit{Monitoramento da Cidade, Pronta Resposta, Gestão de Crises, Operações Integradas, Gestão de Eventos e Prevenção de Problemas.}

Conforme o citado modelo \cite{ModeloGestaoCOP}, a linha de atuação de Monitoramento da Cidade possui três vertentes: o monitoramento de sensores não especialistas, tais como câmeras, o monitoramento de sensores especialistas, tais como os meteorológicos, e o monitoramento de fontes de inteligência. Seu objetivo final é permitir que o COP-BH aja preventivamente a fim de evitar problemas públicos ou responder adequadamente a estes, minorando as suas consequências.

Este trabalho analisa e propõe uma mudança no método de análise de dados, bem como sua automatização, para suporte a uma das atividades da linha de atuação de Monitoramento da Cidade, relacionadas ao monitoramento de câmeras, que se delinea a seguir.

\subsection{Contextualização do Projeto}

Conforme descreve o modelo \cite{ModeloGestaoCOP}, a atividade de monitoramento se utiliza de diversos tipos de sensores para sua consecução. Dentre estes sensores, as câmeras instaladas em locais públicos permitem acesso à visualização de diversos pontos da cidade em tempo real.

A atividade de monitoramento das câmeras, ou seja, a visualização das suas imagens em tempo real, é realizada por agentes das Instituições que têm assento no ambiente chamado Sala de Controle Integrado, local onde as imagens são acessadas com o propósito de identificar problemas públicos mais rapidamente, a fim de possibilitar maior celeridade também à resposta.

São diversos os tipos de problemas públicos monitorados, tais como a deposição clandestina de lixo ou inservíveis em locais impróprios, animais de grande porte soltos em via pública, pichação, depredação, acidentes de trânsito, invasão de áreas protegidas e furtos de cabos, entre outros.

Para tornar o serviço mais eficiente, o COP-BH desenvolveu um processo de análise de dados sobre problemas públicos para elaborar um roteiro de monitoramento, que sugere os problemas que devem ser monitorados, os períodos de maior relevância para executar o monitoramento e quais as câmeras deveriam ser monitoradas. 

A partir deste roteiro, as equipes de monitoramento teriam um guia de referência que poderia aprimorar o seu trabalho no sentido da eficiência -- pois conheceriam as câmeras deveriam priorizar no monitoramento e quando deveriam ser monitoradas, e da eficácia -- pois isso poderia aumentar a probabilidade de visualizar algum problema público durante o monitoramento.

Para fins deste trabalho, o problema público analisado é o furto de cabos. Para isso, foram utilizados dados reais de ocorrências públicas deste problema em Belo Horizonte, e são originados do Sistema de Gestão de Ocorrências Integradas (SICOP Ocorrências), por meio do qual as Instituições e Agências compartilham informações no COP-BH, Sistema de Gestão da Operação (SGO) utilizado pela BHTrans para controle de suas operações, e dados de furtos de cabos de telefonia compartilhados pelas operadoras de telefonia Oi, Claro, Vivo e V.tal. Estes dados são os mesmos utilizados na elaboração do roteiro de monitoramento. O dataset utilizado para este trabalho contém 18.590 observações sobre o problema de furto de cabos, no período de 03/2018 a 05/2025, e consiste dos seguintes atributos:

\begin{itemize}
  \item{\texttt{origem}, que apresenta de qual sistema os dados se originaram;}
  \item{\texttt{data\_hora}, que apresenta a data e hora em que o furto ocorreu;}
  \item{\texttt{mes\_ano}, que apresenta o mês e ano da ocorrência (dados adicionados para facilitar a análise);}
  \item{\texttt{trimestre\_ano}, que apresenta o trimestre e ano da ocorrência (dados adicionados para facilitar a análise);}
  \item{\texttt{latitude} e \texttt{longitude}, que apresenta a localização exata da ocorrência; e}
  \item{\texttt{endereco} e \texttt{regional}, que apresenta o endereço físico aproximado da ocorrência.}
\end{itemize}

\subsection{Definição do Problema}

Atualmente, o COP-BH elabora o roteiro de monitoramento agrupando as ocorrências pela sua localização, e utiliza o método de estimativa de densidade de kernel para gerar um mapa de calor geoespacial (\textit{heatmap}) \cite{Wilkinson2009} para demonstrar visualmente as regiões da cidade de Belo Horizonte onde há maior concentração de furtos, utilizando a ferramenta de sistema de informação geográfica QGIS \cite{Qgis}. Então, dados de localização das câmeras públicas da cidade são sobrepostos ao mapa de calor, sendo este o resultado final do roteiro de monitoramento, conforme demonstrado na Figura \ref{fig:mapa_roteiro}.

\begin{figure}[!htb]
  \captionsetup{singlelinecheck=false}
  \centering
  \includegraphics[scale=1,keepaspectratio]{dados/images/mapa_roteiro.png}
  \caption{Roteiro de Monitoramento: Mapa de calor de locais de furtos de cabos em Belo Horizonte, com sobreposição de câmeras públicas.}
  \fonte{Produção própria}
  \label{fig:mapa_roteiro}
\end{figure}

Contudo, a análise de dados agregados, na forma como é feita, pode levar a interpretações incorretas. Como se poderá verificar na Figura \ref{fig:grafico_barras_ocorrencias_ano}, que demonstra a evolução dos eventos de furtos de cabo nos últimos anos, houve um aumento abrupto ocorrido do ano de 2022 para o ano de 2023, devido ao acréscimo de dados das operadoras Claro, Vivo e V.tal. Conquanto haja uma aparente e relativa estabilização do ano de 2023 para 2024, não se pode extrapolar projetando a linha de tendência para o ano de 2025. A regressão linear, ao representar apenas a tendência média, é inerentemente limitada para modelar mudanças bruscas de comportamento, como a ocorrida de 2022 para 2023. 

Essa limitação é um problema bem documentado na literatura estatística, pois a inclinação do modelo pode ser artificialmente influenciada, mascarando a verdadeira natureza dos dados e novos padrões de comportamento, conforme estabeleceu \cite{Anscombe1973}. Além disso, conforme \cite{Gujarati2011}, ``mudanças estruturais'' nos dados, como a ocorrida no dataset, invalidam a suposição de os parâmetros seriam constantes ao longo do tempo, o que é fundamental para um modelo de regressão único.

\begin{figure}[!htb]
  \captionsetup{singlelinecheck=false}
  \centering
  \includegraphics[scale=0.5,keepaspectratio]{dados/images/grafico_barras_ocorrencias_ano.png}
  \caption{Evolução dos eventos de furtos de cabo em Belo Horizonte, no período de 03/2018 a 05/2025.}
  \fonte{Produção própria}
  \label{fig:grafico_barras_ocorrencias_ano}
\end{figure}

Além disso, o simples incremento de novas observações ao conjunto de dados pode ocultar ou impedir a captura de alterações e mudanças, que embora possam parecer sutis, são de fundamental importância para a eficácia da análise, tais como alterações na localização das concentrações do fenômeno e sazonalidades. 

Conforme concluiu Robinson \cite{Robinson1950}, erros podem ser cometidos, ou conclusões equivocadas podem ser obtidas ao se inferir comportamentos individuais a partir de estatísticas de dados agregados, a chamada ``falácia ecológica'' (\textit{``Ecological Fallacy''}). Seu estudo estabeleceu formalmente o problema, mostrando como correlações em nível de grupo (estados, cidades, por exemplo) podem ser completamente diferentes ou até opostas às correlações em nível individual.

Por sua vez, Weisburd \cite{Weisburd2015} complementou este entendimento, ao argumentar pela necessidade de ir além da análise de dados agregados, e buscar padrões espaciais por meio de \textit{hotspots}, ou pontos quentes, onde há concentração de crimes. 

Acrescenta-se, ainda, o efeito de deslocamento geográfico (\textit{``displacement''}) conceituado por Barr \cite{Barr1990}, que explorou as diferentes formas como o crime se move em resposta a ações de prevenção. As figuras apresentadas na Tabela \ref{tab:tabela_mapas} representam, empiricamente, esse deslocamento espacial dos eventos de furto de cabo em Belo Horizonte, além de indicar as diferentes concentrações dos eventos no tempo. Para fins deste comparativo foram considerados apenas os anos de 2023 e 2024, por serem os anos em que há maior volume de dados no dataset.

\begin{table}[htbp]
  \centering
  \begin{tabular}{|m{3.5cm}|m{3.5cm}|m{3.5cm}|m{3.5cm}|}
    \hline
      \begin{minipage}{\linewidth}
        \centering
        \includegraphics[width=3.5cm]{dados/images/mapa_20231_2.png}
        \captionof{figure}{1T 2023}
      \end{minipage}
      &
      \begin{minipage}{\linewidth}
        \centering
        \includegraphics[width=3.5cm]{dados/images/mapa_20232_2.png}
        \captionof{figure}{2T 2023}
      \end{minipage}
      &
      \begin{minipage}{\linewidth}
        \centering
        \includegraphics[width=3.5cm]{dados/images/mapa_20233_2.png}
        \captionof{figure}{3T 2023}
      \end{minipage}
      &
      \begin{minipage}{\linewidth}
        \centering
        \includegraphics[width=3.5cm]{dados/images/mapa_20234_2.png}
        \captionof{figure}{4T 2023}
      \end{minipage}
    \\ \hline
      \begin{minipage}{\linewidth}
        \centering
        \includegraphics[width=3.5cm]{dados/images/mapa_20241_2.png}
        \captionof{figure}{1T 2024}
      \end{minipage}
      &
      \begin{minipage}{\linewidth}
        \centering
        \includegraphics[width=3.5cm]{dados/images/mapa_20242_2.png}
        \captionof{figure}{2T 2024}
      \end{minipage}
      &
      \begin{minipage}{\linewidth}
        \centering
        \includegraphics[width=3.5cm]{dados/images/mapa_20243_2.png}
        \captionof{figure}{3T 2024}
      \end{minipage}
      &
      \begin{minipage}{\linewidth}
        \centering
        \includegraphics[width=3.5cm]{dados/images/mapa_20244_2.png}
        \captionof{figure}{4T 2024}
      \end{minipage}
    \\ \hline
  \end{tabular}
  \caption{Comparação dos mapas de calor por densidade de kernel das observações de furtos de cabo em Belo Horizonte, dos anos de 2023 e 2024, por Trimestre.}
  \fonte{Produção própria}
  \label{tab:tabela_mapas}
\end{table}

Estes fatos corroboram o entendimento de que o método atualmente utilizado pelo COP-BH ainda é insuficiente para uma análise adequada do fenômeno, além do fato de ser executado manualmente. Assim, esse trabalho busca propor um novo método de análise, bem como uma ferramenta para automatizar este processo.

\subsection{Relevância do Problema}

A relevância do problema estudado pode ser verificada na melhoria da atividade de monitoramento realizada no COP-BH. Salienta-se que o monitoramento é realizado sob a perspectiva da prevenção e pronta resposta à ocorrência de crimes, e nesse contexto o crime de furto de cabos. Esse crime tem diversas consequências negativas, tanto econômicas quanto sociais.

Do ponto de vista social e comunitário, podemos citar diversos problemas que o furto de cabos proporcionam à população em geral, tais como a interrupção de serviços de água, luz, acesso à Internet, serviços de saúde, e transtornos nos sistemas de tráfego e transporte público da cidade  (\citenum{Cemig}, \citenum{OGlobo}, \citenum{R7}, \citenum{OTempoA}, \citenum{OTempoB} e \citenum{HojeEmDia}).  

Além disso, há também o impacto econômico, pois o restabelecimento destes serviços demanda o gasto com reparos na infraestrutura de rede elétrica, de telefonia ou de dados, seja pela Administração Pública ou por Concessionárias (\citenum{EstadoDeMinasB}). Pequenos comerciantes e prestadores de serviços também sofrem impactos econômicos decorrentes da interrupção destes serviços essenciais ao funcionamento dos seus estabelecimentos (\citenum{OTempoB}).

Por fim, citam-se os riscos à integridade física dos próprios cometedores deste crime estão expostos, pois podem ser eletrocutados ao manipular cabos e fios de alta tensão, sem as devidas proteções, ou sofrer quedas de altura, ao subir em postes sem equipamentos próprios para isso (\citenum{EstadoDeMinas}). 

Portanto, pode-se perceber que o furto de cabos é um problema público relevante, e que no âmbito do COP-BH é enfrentado por meio do monitoramento por câmeras e correspondente pronta resposta. Neste sentido, ao buscar o aprimoramento da análise de dados em subsídio ao monitoramento inteligente, este trabalho se propõe a complementar o apoio à resolução de um problema relevante, e que atinge a diversas cidades brasileiras.

\subsection{Justificativa}



\subsection{Desafios do Projeto}

\subsection{Contribuição}

%-----------------------------------------------------------------------------------------

\section{OBJETIVOS}
\label{sec:objetivos}
% (máximo de 1/2 página)

\subsection{Objetivo Geral}
\label{subsec:objgeral}
% Objetivo Geral--------------------------------------------------------------------------
%-----------------------------------------------------------------------------------------

\subsection{Objetivos Específicos}
\label{subsec:objespc}
% Objetivos Específicos-------------------------------------------------------------------
%-----------------------------------------------------------------------------------------

\section{TRABALHOS CORRELATOS}
\label{sec:estadoarte}
% Estado da Arte--------------------------------------------------------------------------
% (máximo de 2 páginas)
%-----------------------------------------------------------------------------------------

\section{METODOLOGIA/METODOLOGIA DE DESENVOLVIMENTO} % Escolher o nome mais adequado ao trabalho
\label{sec:metodologia}
% Procedimentos Metodológicos/Metodologia-------------------------------------------------
% (máximo de 2 páginas)
%-----------------------------------------------------------------------------------------

\section{CONCLUSÃO/CONSIDERAÇÕES FINAIS} % Escolher o nome mais adequado ao trabalho
\label{sec:conclusao}
% Conclusão/Considerações Finais----------------------------------------------------------
% (máximo de ½ página)
%-----------------------------------------------------------------------------------------

\section{PLANEJAMENTO DO TRABALHO}
\label{sec:planejamento}
% Planejamento do Trabalho----------------------------------------------------------------
% Esta seção não precisa ser editada, apenas edite o quadro 1 armazenada no diretório ".\dados\quadros"
\input{./dados/quadros/quadro1}

\subsection{DA PROPOSTA AO PROJETO}
%-----------------------------------------------------------------------------------------

\section{RECURSOS NECESSÁRIOS}
\label{sec:recursos}
% Recursos Necessários--------------------------------------------------------------------
%-----------------------------------------------------------------------------------------

\section{HORÁRIO DE TRABALHO}
\label{sec:horário}
% Horário de Trabalho---------------------------------------------------------------------
% Esta seção não precisa ser editada, apenas edite o quadro 2 armazenada no diretório ".\dados\quadros"
\input{./dados/quadros/quadro2}
%-----------------------------------------------------------------------------------------
