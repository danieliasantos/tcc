% PROPOSTA DE TRABALHO DE CONCLUSÃO DE CURSO-----------------------------------------------------------

\chapter{PROPOSTA DE TRABALHO DE CONCLUSÃO DE CURSO}
\label{chap:proposta}

\section{TÍTULO}
\label{sec:titulo}
% Informe o título do trabalho-------------------------------------------------------------------------
%------------------------------------------------------------------------------------------------------
\thetitle.

\section{MODALIDADE DO TRABALHO}
\label{sec:modalidade}
% Indique a Modalidade do Trabalho---------------------------------------------------------------------
% Opções:
% - Pesquisa
% - Desenvolvimento de Sistemas
Pesquisa e Desenvolvimento de Sistemas
%------------------------------------------------------------------------------------------------------

\section{ÁREA DO TRABALHO}
\label{sec:area}
% Indique a Área do Trabalho---------------------------------------------------------------------------
Ciência de dados.
%------------------------------------------------------------------------------------------------------

\section{RESUMO}
\label{sec:resumo}
% Resumo do Trabalho-----------------------------------------------------------------------------------
% (máximo de 200 palavras)
Métodos tradicionais de mapeamento de hotspots, como a Estimativa de Densidade por Kernel (KDE) estática, são insuficientes para analisar fenômenos criminais dinâmicos, pois ocultam a evolução dos padrões no tempo e espaço. Este trabalho tem o objetivo de desenvolver e validar uma metodologia computacional automatizada para a análise espaço-temporal dinâmica de hotspots de furto de cabos, a fim de aprimorar a atividade de monitoramento do Centro Integrado de Operações da Prefeitura de Belo Horizonte (COP-BH). A metodologia proposta, aplicada sobre um dataset real, segmenta a análise em janelas temporais para identificar clusters com significância estatística. A validação será realizada por meio da comparação do novo método com a abordagem atualmente utilizada, através do Índice de Acurácia Preditiva (PAI). A partir disso pretende-se identificar o surgimento, dissipação e deslocamento dos hotspots, padrões que a análise estática não detecta. Espera-se que a metodologia ofereça uma ferramenta de inteligência mais eficaz para as ações de monitoramento, permitindo uma alocação eficiente de recursos, além de estabelecer uma base técnica robusta para aplicações futuras, como a ativação seletiva de analíticos de vídeo para prevenção.

Palavras-chave: Análise de dados, Mapeamento de Hotspots, Análise Espaço-Temporal, Ações Baseadas em Evidências, Furto de Cabos.

%------------------------------------------------------------------------------------------------------

