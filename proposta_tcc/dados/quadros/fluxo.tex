\begin{figure}[htbp]
    \centering
    \begin{tikzpicture}[
        % Distâncias: 1cm na vertical (dentro dos blocos), 4cm na horizontal (entre os blocos)
        node distance=1cm and 2cm,
        process/.style={
            rectangle, 
            rounded corners, 
            draw, 
            text centered, 
            text width=3.4cm, 
            minimum height=1.2cm,
            font=\footnotesize
        },
        decision/.style={
            diamond, 
            draw, 
            aspect=1.8, 
            text centered, 
            text width=2.8cm,
            font=\footnotesize
        },
        terminator/.style={
            ellipse, 
            draw, 
            fill=gray!20,
            font=\footnotesize
        },
        fitbox/.style={
            draw, 
            rounded corners, 
            inner ysep=0.5cm, 
            inner xsep=0.3cm,
            label={[font=\footnotesize\bfseries]above:#1}
        },
        arrow/.style={
            thick, 
            ->, 
            >=Stealth
        }
    ]

    % --- LINHA SUPERIOR (Fases 1, 2, 3) ---

    % Fase 2 (Centro - Âncora principal)
    \node (D) [process] {Definição da Janela Temporal de Análise};
    \node (E) [process, below=of D] {Aplicação do Método de Detecção de Hotspots};% \\ \tiny\textit{(ex: KSS)}};
    \node (F) [process, below=of E] {Desenvolvimento da Análise de Dinâmica};% \\ \tiny\textit{(Comparação entre janelas)}};

    % Fase 1 (À Esquerda da Fase 2)
    \node (A) [process, left=of D] {Revisão Bibliográfica};
    \node (B) [process, below=of A] {Análise do Processo Atual do COP-BH};
    \node (C) [process, below=of B] {Pré-processamento e AED};

    % Fase 3 (À Direita da Fase 2)
    \node (G) [process, right=of D] {Definição da Arquitetura e Tecnologias};
    \node (H) [process, below=of G] {Codificação da Ferramenta Automatizada};
    \node (I) [process, below=of H] {Geração de Relatórios e Mapas Visuais};


    % --- BLOCOS INFERIORES ---

    % Fase 4 (Abaixo das Fases 2 e 3, com maior espaçamento vertical)
    \node (J) [process, below=of F, node distance=4cm, xshift=3cm, yshift=-2cm] {Definição de Métricas};% \\ \tiny\textit{(ex: PAI)}};
    \node (K) [process, below=of J] {Execução de Testes Comparativos};% \\ \tiny\textit{(Novo vs. Tradicional)}};
    \node (L) [process, below=of K] {Análise dos Resultados};% \\ \tiny\textit{(Quantitativos e Qualitativos)}};

    % Conclusão (Abaixo das Fases 1 e 2, e à esquerda da Fase 4)
    \node (M) [decision, left=of K] {Resultados Satisfatórios?};
    \node (N) [process, below=of M] {Análise Final e Escrita da Dissertação/Tese};
    \node (O) [terminator, below=of N, node distance=1cm] {Fim};
    
    
    % --- AGRUPAMENTO DAS FASES (CAIXAS) ---
    \node [fitbox={1: Diag. e AED}, fit=(A)(B)(C)] {};
    \node [fitbox={2: Modelagem}, fit=(D)(E)(F)] {};
    \node [fitbox={3: Implementação}, fit=(G)(H)(I)] {};
    \node [fitbox={4: Validação}, fit=(J)(K)(L)] {};
    \node [fitbox={Conclusão}, fit=(M)(N)(O)] {};
    
    
    % --- CONEXÕES (SETAS) ---
    % Fluxo interno dos blocos
    \path [arrow] (A) edge (B);
    \path [arrow] (B) edge (C);
    \path [arrow] (D) edge (E);
    \path [arrow] (E) edge (F);
    \path [arrow] (G) edge (H);
    \path [arrow] (H) edge (I);
    \path [arrow] (J) edge (K);
    \path [arrow] (K) edge (L);
    \path [arrow] (M) edge node[right, font=\tiny] {Sim} (N);
    \path [arrow] (N) edge (O);
    
    % Conexões curvadas entre as fases
    \draw [arrow] (C.east) to[out=-30, in=160, looseness=0.8] node[above, sloped] {} (D.west); % Fase 1 -> Fase 2
    \draw [arrow] (F.east) to[out=-30, in=160, looseness=0.8] node[above, sloped] {} (G.west); % Fase 2 -> Fase 3;
    \draw [arrow] (I.south) to[out=-60, in=-340, looseness=1] node[right, sloped] {} (J.east);   % Fase 3 -> Fase 4
    \draw [arrow] (L.west) to[out=60, in=-350, looseness=0.8] node[above, sloped] {} (M.east); % Fase 4 -> Conclusão
    
    % Seta do loop (Não)
    \draw [arrow, dashed] (M.north) to[out=135, in=-90, looseness=0.8] node[above, sloped] {Não} (F.south);

    \end{tikzpicture}
    \caption{Fluxograma da metodologia de pesquisa proposta (Layout Final Ajustado).}
    \label{fig:fluxograma_metodologia_final}
\end{figure}