\begin{figure}[htbp]
    \centering
    \begin{tikzpicture}[
        % Distâncias: 1cm na vertical (dentro dos blocos), 4cm na horizontal (entre os blocos)
        node distance=1cm and 2cm,
        process/.style={
            rectangle, 
            rounded corners, 
            draw, 
            text centered, 
            text width=3.4cm, 
            minimum height=1.2cm,
            font=\footnotesize
        },
        decision/.style={
            diamond, 
            draw, 
            aspect=1.8, 
            text centered, 
            text width=2.8cm,
            font=\footnotesize
        },
        terminator/.style={
            ellipse, 
            draw, 
            fill=gray!20,
            font=\footnotesize
        },
        fitbox/.style={
            draw, 
            rounded corners, 
            inner ysep=0.5cm, 
            inner xsep=0.3cm,
            label={[font=\footnotesize\bfseries]above:#1}
        },
        arrow/.style={
            thick, 
            ->, 
            >=Stealth
        }
    ]

    % --- LINHA SUPERIOR (Fases 1, 2) ---

    % Fase 2 (Centro - Âncora principal)
    \node (D) [process] {Definição da unidade temporal de análise };
    \node (E) [process, below=of D] {Desenvolvimento do modelo de detecção de hotspots};
    \node (F) [process, below=of E] {Desenvolvimento da análise dinâmica};

    % Fase 1 (À Esquerda da Fase 2)
    \node (A) [process, left=of D] {Revisão bibliográfica};
    \node (B) [process, below=of A] {Análise do processo atual do COP-BH};
    \node (C) [process, below=of B] {Análise dos dados};

    % --- BLOCOS INFERIORES ---

    % Fase 3 (Abaixo da Fase 3)
    \node (G) [process, below=of F, yshift=-1cm] {Definição da arquitetura};
    \node (H) [process, below=of G] {Definição das tecnologias utilizadas};
    \node (I) [process, below=of H] {Desenvolvimento da automatização do fluxo};

    % Fase 4 (À esquerda da Fase 3)
    \node (J) [process, left=of G] {Definiçãoda métrica de avaliação};
    \node (K) [process, below=of J] {Análise e comparação visual};
    \node (L) [process, below=of K] {Análise dos Resultados};

    %\node (M) [terminator, left=of M, node distance=5cm] {Fim};
    \node (M) [terminator, left=of L] {Fim};
    
    
    % --- AGRUPAMENTO DAS FASES (CAIXAS) ---
    \node [fitbox={1: Análise de Dados}, fit=(A)(B)(C)] {};
    \node [fitbox={2: Modelagem}, fit=(D)(E)(F)] {};
    \node [fitbox={3: Implementação}, fit=(G)(H)(I)] {};
    \node [fitbox={4: Validação}, fit=(J)(K)(L)] {};
    
    
    % --- CONEXÕES (SETAS) ---
    \path [arrow] (A) edge (B);
    \path [arrow] (B) edge (C);
    \draw [arrow] (C.east) to[out=-30, in=160, looseness=0.8] node[above, sloped] {} (D.west); % Fase 1 -> Fase 2
    \path [arrow] (D) edge (E);
    \path [arrow] (E) edge (F);
    \draw [arrow] (F.east) to[out=-40, in=-350, looseness=0.8] node[above, sloped] {} (G.east); % Fase 2 -> Fase 3;
    \path [arrow] (G) edge (H);
    \path [arrow] (H) edge (I);
    \draw [arrow] (I.west) to[out=-160, in=-350, looseness=0.8] node[right, sloped] {} (J.east); % Fase 3 -> Fase 4
    \path [arrow] (J) edge (K);
    \path [arrow] (K) edge (L);
    \draw [arrow] (L.west) to[out=-180, in=-350, looseness=0.8] node[right, sloped] {} (M.east); % Fase 4 -> Conclusão
    
    \end{tikzpicture}
    \caption{Fluxograma da metodologia proposta de pesquisa.}
    \label{fig:fluxograma_metodologia_final}
\end{figure}